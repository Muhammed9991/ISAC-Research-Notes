\documentclass[12pt, a4paper]{article}
\usepackage[utf8]{inputenc}
\usepackage{amsmath, amssymb}
\usepackage{geometry}
\usepackage{enumitem}

% Setup margins
\geometry{margin=1in}

\title{\textbf{Secure Null-Steering in mmWave ISAC Systems}}
\author{}
\date{}

\begin{document}

\maketitle

\section*{1. System Overview}
Consider a downlink Integrated Sensing and Communications (ISAC) system where a Base Station (``Alice'') transmits data to a single legitimate user (``Bob'') while simultaneously tracking potential eavesdroppers (``Eve'') using radar echoes.

Your goal is to design a transit beamformer that maximises power to Bob while suppressing the signal leakage to Eve (Null-Steering). However, you must first determine if the physical limitations of the antenna array allow for this spatial separation.

\section*{2. System Parameters}
\begin{itemize}
    \item \textbf{Carrier Frequency ($f_c$):} 28 GHz.
    \item \textbf{Transmitter (Alice):}
    \begin{itemize}
        \item \textbf{Antenna Array:} A massive MIMO Uniform Planar Array (UPA) with \textbf{64 elements}.
        \item \textbf{Geometry:} Arranged in an $\mathbf{8 \times 8}$ \textbf{grid} in the \textbf{Y-Z plane} (centered at the origin $[0,0,0]$).
        \item \textbf{Spacing:} Elements are spaced by half-wavelength ($\lambda/2$) in both horizontal ($y$) and vertical ($z$) directions.
        \item \textbf{Sensing Zone:} Alice can detect targets within a $120^\circ$ sector ($\pm 60^\circ$) in azimuth and elevation.
    \end{itemize}
    \item \textbf{Channel State Information (CSI):} Alice has perfect knowledge of the instantaneous channels for both Bob and Eve.
\end{itemize}

\section*{3. Scenario Coordinates}
All coordinates are given in meters $(x, y, z)$ relative to Alice at $(0,0,0)$.
The array faces the positive x-axis.

\begin{itemize}
    \item \textbf{Legitimate User (Bob):}
    \begin{itemize}
        \item \textbf{Position:} Fixed at $P_{Bob} = (30, 10, 5)$.
    \end{itemize}

    \item \textbf{Eavesdropper (Eve):}
    \begin{itemize}
        \item \textbf{Case A:} $P_{Eve_A} = (30, -20, 5)$
    \end{itemize}
\end{itemize}

\section*{4. Signal Model}
To solve the problem, assume a Line-of-Sight (LoS) channel model. The channel vector $h$ for a user at a specific angle is defined by the array steering vector $a(\theta, \phi)$.

For a UPA in the Y-Z plane with $N_y$ columns and $N_z$ rows ($N_y=8, N_z=8$):

\begin{equation}
    a(\theta, \phi) = a_z(\phi) \otimes a_y(\theta, \phi)\label{eq:equation}
\end{equation}

Where $\otimes$ denotes the Kronecker product. The steering vectors for the y and z dimensions are:

\begin{equation}
    [a\_y(\theta, \phi)]_n = e^{-j \pi n \sin(\theta) \cos(\phi)}, \quad n = 0, \dots, N_y-1\label{eq:equation2}
\end{equation}

\begin{equation}
    [a\_z(\phi)]_m = e^{-j \pi m \sin(\phi)}, \quad m = 0, \dots, N_z-1\label{eq:equation3}
\end{equation}

\begin{itemize}
    \item $\theta$: Azimuth angle (horizontal angle from the x-axis).
    \item $\phi$: Elevation angle (vertical angle from the x-y plane).
\end{itemize}

\section*{5. Problem Questions}

\subsection*{Part 1: The Rayleigh Resolution Limit}
Before attempting to null Eve, you must verify if she is spatially distinguishable from Bob.

\begin{enumerate}
    \item Calculate the \textbf{Rayleigh Resolution Limit ($\Delta \theta_{res}$)} for this $8 \times 8$ array at 28 GHz.
    \emph{(Hint: The resolution limit is determined by the array aperture size ($D$) and wavelength ($\lambda$).)}
\end{enumerate}

\subsection*{Part 2: Zero-Forcing Beamforming Derivation}
For the resolvable scenarios, we wish to construct a beamforming weight vector $w \in \mathbb{C}^{64 \times 1}$.

\begin{enumerate}
    \item Formulate the optimisation problem to find $w$ such that:
    \begin{itemize}
        \item The gain towards Bob is normalised to 1: $|w^H h_{Bob}| = 1$
        \item The power towards Eve is forced to zero: $|w^H h_{Eve}| = 0$
    \end{itemize}
    \item Derive the closed-form mathematical expression for $w$ using the \textbf{Zero-Forcing (ZF)} or \textbf{Projection} method.
\end{enumerate}

\subsection*{Part 3: Numerical Calculation}
\begin{enumerate}
    \item Convert the Cartesian coordinates for Bob and \textbf{Case A} ($P_{Eve_A}$) into Azimuth ($\theta$) and Elevation ($\phi$) angles.
    \item Using your derived expression from Part 2, calculate the beamforming weight vector $w$ for \textbf{Case A}.
\end{enumerate}

\newpage
\section*{Solution}\label{sec:solution}
    \subsection*{Deriving the Zero-Forcing Beamforming Weight Vector:}
    Constrained optimisation problem

    \begin{equation} \label{eq:equation4}
        \text{minimise } \|w - h_{Bob}\|^2 \quad \text{subject to } w^H h_{Eve} = 0
    \end{equation}

    Using Lagrange multipliers, we form the Lagrangian (complex lagrangian chosen as we are dealing with complex vectors):

    \begin{equation} \label{eq:equation5}
        \begin{split}
            \mathcal{L}(w, \lambda) &= \|w - h_{Bob}\|^2 + \lambda (w^H h_{Eve}) \\
                                &= (w - h_{Bob})^{H} \cdot (w - h_{Bob}) + \lambda (w^H h_{Eve})
        \end{split}
    \end{equation}

    Finding the Wirtinger derivative:
    \begin{equation} \label{eq:equation6}
            \frac{\partial \mathcal{L}}{\partial w^{*}} \&= (w - h_{Bob}) + \lambda h_{Eve} = 0
    \end{equation}

    Making $w$ the subject and multiply by $h_{Eve}^H$:
    \begin{equation} \label{eq:equation7}
        w = h_{Bob} - \lambda h_{Eve} \implies h_{Eve}^H w = h_{Eve}^H h_{Bob} - \lambda (h_{Eve}^H h_{Eve}) = 0
    \end{equation}

    Making $\lambda$ the subject:
    \begin{equation} \label{eq:equation8}
        \lambda = \frac{h_{Eve}^H h_{Bob}}{\|h_{Eve}\|^2}
    \end{equation}

    Equation~\ref{eq:equation8} substituted into Equation \ref{eq:equation7} and factoring out $h_{Bob}$ gives us the Zero-Forcing Weight Vector
    \begin{equation} \label{eq:equation9}
        w = h_{Bob} - \frac{h_{Eve}^H h_{Bob}}{\|h_{Eve}\|^2} h_{Eve} = \left( I - \underbrace{\frac{h_{Eve} h_{Eve}^H}{\|h_{Eve}\|^2}}_{\text{Spatial Coefficient}} \right) h_{Bob}
    \end{equation}

    \subsection*{Deriving Normalised Spatial Correlation}
    Below is the spatial correlation coefficient used in Equation \ref{eq:equation9}.

    \begin{equation} \label{eq:equation10}
        \frac{h_{Eve} h_{Eve}^H}{\|h_{Eve}\|^2}
    \end{equation}
    The squared norm is given by $\|h_{Eve}\|^2 = \sum_{n=1}^{64} |h_{n}|^{2}$. \\

    Assuming $h_{n}$ represents a phase-shift element of the form $e^{-j\theta_n}$, its magnitude is $|e^{-j\theta_n}| = 1$.
    Therefore:
    \begin{equation} \label{eq:equation11}
        \|h_{Eve}\|^2 = \sum_{n=1}^{64} 1 = 64 \leftarrow N_{Total}
    \end{equation}

    Normalised Spatial Correlation:
    \begin{equation} \label{eq:equation12}
       \psi = \frac{h_{Eve}^H h_{Bob}}{N_{Total}}
    \end{equation}

    \subsection*{Array Normalisation Factor}
    To normalise received signal by magintude 1 (Unity gain).
    \begin{equation} \label{eq:equation13}
        w^{H} h_{Bob} = 1
    \end{equation}

    Assuming $w$ is a scaled version of $h_{Bob}$
    \begin{equation} \label{eq:equation14}
        w = \alpha h_{Bob}
    \end{equation}

    Substituting Equation \ref{eq:equation14} into Equation \ref{eq:equation13}:
    \begin{equation} \label{eq:equation15}
        \begin{split}
            (\alpha h_{Bob})^{H} h_{Bob} &= 1 \\
             \alpha \underbrace{(h_{Bob}^{H} h_{Bob})}_{64} &= 1 \\
            \alpha \cdot 64 &= 1 \\
            \alpha &= \frac{1}{64}
        \end{split}
    \end{equation}

    \subsection*{Final Beamforming Weight Vector}
    Combining all parts together, the final beamforming weight vector $w_{Final}$ is given by:
    \begin{equation} \label{eq:equation16}
        \begin{split}
            w_{Final}   &= \alpha (h_{Bob} - \psi h_{Eve}) \\
                &= \alpha h_{Bob} - (\alpha \psi) h_{Eve} \quad \text{let } \beta = \alpha \psi
        \end{split}
    \end{equation}

    Crucially, $\psi$ is a complex number (magnitude and phase). For this scenario:
    $$ \alpha \approx \frac{1}{64} \approx 0.0159 $$
    $$ \psi \approx 0.121 - j0.032 $$
    $$ \beta = \alpha \psi \approx 0.0019 - j0.0005 $$

    Thus:
    $$ w_{Final} = 0.0159 h_{Bob} - (0.0019 - j0.0005) h_{Eve} $$

    \newpage
    \subsection*{Formulas}

    \begin{itemize}
        \item Azimuth Angle: $\theta = \tan^{-1}\left(\frac{y}{x}\right)$
        \item Elevation Angle: $\phi = \tan^{-1}\left(\frac{z}{\sqrt{x^2 + y^2}}\right)$
    \end{itemize}

    \subsection*{Case A Calculation}

    \underline{Bob}
    \begin{equation} \label{eq:equation17}
        \begin{split}
            \theta_{Bob} &= \tan^{-1}\left(\frac{10}{30}\right) \approx 18.43^\circ \\
            % Corrected: Denominator must include y^2 (sqrt(30^2 + 10^2) = 31.62)
            \phi_{Bob} &= \tan^{-1}\left(\frac{5}{31.62}\right) \approx 8.99^\circ
        \end{split}
    \end{equation}

    Steering vector calculation for Bob (first element $n=0$ is 1):
    \begin{equation} \label{eq:equationBobSteering}
        [h_{Bob}]_0 = 1
    \end{equation}

    \underline{Eve A}
    \begin{equation} \label{eq:equation18}
        \begin{split}
            \theta_{Eve_A} &= \tan^{-1}\left(\frac{-20}{30}\right) \approx -33.69^\circ \\
            % Corrected: Denominator must include y^2 (sqrt(30^2 + (-20)^2) = 36.06)
            \phi_{Eve_A} &= \tan^{-1}\left(\frac{5}{36.06}\right) \approx 7.90^\circ
        \end{split}
    \end{equation}

    Steering vector calculation for Eve (first element $n=0$ is 1):
    \begin{equation} \label{eq:equationEveSteering}
        [h_{Eve_A}]_0 = 1
    \end{equation}

    \underline{For Antenna 1 (index 0)}
    Using the complex weights derived in Equation \ref{eq:equation16}:
    \begin{equation} \label{eq:equation19}
        \begin{split}
            w_{0} &= \alpha \cdot [h_{Bob}]_0 - \beta \cdot [h_{Eve_A}]_0 \\
                  &= 0.0159 \cdot (1) - (0.0019 - j0.0005) \cdot (1) \\
                  &= 0.0140 + j0.0005
        \end{split}
    \end{equation}

    This complex weight ensures the signal phase cancels out perfectly at Eve's location.

    The aforementioned procedure can be repeated for all 64 antenna elements to construct the complete beamforming weight vector $w$.

\end{document}
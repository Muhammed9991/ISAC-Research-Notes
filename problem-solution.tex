\documentclass[12pt, a4paper]{article}
\usepackage[utf8]{inputenc}
\usepackage{amsmath, amssymb}
\usepackage{geometry}
\usepackage{enumitem}

% Setup margins
\geometry{margin=1in}

\title{\textbf{Secure Null-Steering in mmWave ISAC Systems}}
\author{}
\date{}

\begin{document}

\maketitle

\section*{1. System Overview}
Consider a downlink Integrated Sensing and Communications (ISAC) system where a Base Station (``Alice'') transmits data to a single legitimate user (``Bob'') while simultaneously tracking potential eavesdroppers (``Eve'') using radar echoes.
Your goal is to design a transit beamformer that maximises power to Bob while suppressing the signal leakage to Eve (Null-Steering).
However, you must first determine if the physical limitations of the antenna array allow for this spatial separation.

\section*{2. System Parameters}
\begin{itemize}
    \item \textbf{Carrier Frequency ($f_c$):} 28 GHz.
    \item \textbf{Transmitter (Alice):}
    \begin{itemize}
        \item \textbf{Antenna Array:} A massive MIMO Uniform Planar Array (UPA) with \textbf{64 elements}.
        \item \textbf{Geometry:} Arranged in an $\mathbf{8 \times 8}$ \textbf{grid} in the \textbf{Y-Z plane} (centered at the origin $[0,0,0]$).
        \item \textbf{Spacing:} Elements are spaced by half-wavelength ($\lambda/2$) in both horizontal ($y$) and vertical ($z$) directions.
        \item \textbf{Sensing Zone:} Alice can detect targets within a $120^\circ$ sector ($\pm 60^\circ$) in azimuth and elevation.
    \end{itemize}
    \item \textbf{Channel State Information (CSI):} Alice has perfect knowledge of the instantaneous channels for both Bob and Eve.
\end{itemize}

\section*{3. Scenario Coordinates}
All coordinates are given in meters $(x, y, z)$ relative to Alice at $(0,0,0)$.
The array faces the positive x-axis.

\begin{itemize}
    \item \textbf{Legitimate User (Bob):}
    \begin{itemize}
        \item \textbf{Position:} Fixed at $P_{Bob} = (30, 10, 5)$.
    \end{itemize}

    \item \textbf{Eavesdropper (Eve):}
    \begin{itemize}
        \item \textbf{Case A:} $P_{Eve_A} = (30, -20, 5)$.
    \end{itemize}
\end{itemize}

\section*{4. Signal Model}
To solve the problem, assume a Line-of-Sight (LoS) channel model.
The channel vector $h$ for a user at a specific angle is defined by the array steering vector $a(\theta, \phi)$.
For a UPA in the Y-Z plane with $N_y$ columns and $N_z$ rows ($N_y=8, N_z=8$):

\begin{equation} \label{eq:equation1}
    a(\theta, \phi) = a_z(\phi) \otimes a_y(\theta, \phi)
\end{equation}

Where $\otimes$ denotes the Kronecker product.
The steering vectors for the y and z dimensions are:

\begin{equation} \label{eq:equation2}
    [a_y(\theta, \phi)]_n = e^{-j \pi n \sin(\theta) \cos(\phi)}, \quad n = 0, \dots, N_y-1
\end{equation}

\begin{equation} \label{eq:equation3}
    [a_z(\phi)]_m = e^{-j \pi m \sin(\phi)}, \quad m = 0, \dots, N_z-1
\end{equation}

\begin{itemize}
    \item $\theta$: Azimuth angle (horizontal angle from the x-axis).
    \item $\phi$ Elevation angle (vertical angle from the x-y plane).
\end{itemize}

\section*{5. Problem Questions}

\subsection*{Part 1: The Rayleigh Resolution Limit}
Before attempting to null Eve, you must verify if she is spatially distinguishable from Bob.
\begin{enumerate}
    \item Calculate the \textbf{Rayleigh Resolution Limit ($\Delta \theta_{res}$)} for this $8 \times 8$ array at 28 GHz.
    \emph{(Hint: The resolution limit is determined by the array aperture size ($D$) and wavelength ($\lambda$).)}
\end{enumerate}

\subsection*{Part 2: Zero-Forcing Beamforming Derivation}
For the resolvable scenarios, we wish to construct a beamforming weight vector $w \in \mathbb{C}^{64 \times 1}$.
\begin{enumerate}
    \item Formulate the optimisation problem to find $w$ such that:
    \begin{itemize}
        \item The gain towards Bob is normalised to 1: $|w^H h_{Bob}| = 1$
        \item The power towards Eve is forced to zero: $|w^H h_{Eve}| = 0$
    \end{itemize}
    \item Derive the closed-form mathematical expression for $w$ using the \textbf{Zero-Forcing (ZF)} or \textbf{Projection} method.
\end{enumerate}

\subsection*{Part 3: Numerical Calculation}
\begin{enumerate}
    \item Convert the Cartesian coordinates for Bob and \textbf{Case A} ($P_{Eve_A}$) into Azimuth ($\theta$) and Elevation ($\phi$) angles.
    \item Using your derived expression from Part 2, calculate the beamforming weight vector $w$ for \textbf{Case A}.
\end{enumerate}

\newpage
\section*{Solution}\label{sec:solution}
    \subsection*{Deriving the Zero-Forcing Beamforming Weight Vector}
    Constrained optimisation problem:

    \begin{equation} \label{eq:equation4}
        \text{minimise } \|w - h_{Bob}\|^2 \quad \text{subject to } w^H h_{Eve} = 0
    \end{equation}

    Using Lagrange multipliers, we form the Lagrangian:

    \begin{equation} \label{eq:equation5}
        \begin{split}
            \mathcal{L}(w, \lambda) &= \|w - h_{Bob}\|^2 + \lambda (w^H h_{Eve}) \\
                                  &= (w - h_{Bob})^{H} (w - h_{Bob}) + \lambda (w^H h_{Eve})
        \end{split}
    \end{equation}

    Finding the Wirtinger derivative with respect to $w^*$:
    \begin{equation} \label{eq:equation6}
            \frac{\partial \mathcal{L}}{\partial w^{*}} = (w - h_{Bob}) + \lambda h_{Eve} = 0
    \end{equation}

    Rearranging to make $w$ the subject:
    \begin{equation} \label{eq:equation7}
        w = h_{Bob} - \lambda h_{Eve}
    \end{equation}

    To find $\lambda$, we strictly enforce the zero-forcing constraint:
    \begin{equation} \label{eq:equation8}
        h_{Eve}^H w = 0
    \end{equation}

    Substituting Equation~\ref{eq:equation7} into Equation~\ref{eq:equation8}:
    \begin{equation} \label{eq:equation9}
        h_{Eve}^H (h_{Bob} - \lambda h_{Eve}) = 0
    \end{equation}

    Expanding and solving for $\lambda$:
    \begin{equation} \label{eq:equation10}
        \begin{split}
            h_{Eve}^H h_{Bob} - \lambda (h_{Eve}^H h_{Eve}) &= 0 \\
            \lambda \|h_{Eve}\|^2 &= h_{Eve}^H h_{Bob} \\
            \lambda &= \frac{h_{Eve}^H h_{Bob}}{\|h_{Eve}\|^2}
        \end{split}
    \end{equation}

    Substituting the scalar value from Equation~\ref{eq:equation10} back into Equation~\ref{eq:equation7} gives the initial expression:
    \begin{equation} \label{eq:equation11}
        w = h_{Bob} - \left( \frac{h_{Eve}^H h_{Bob}}{\|h_{Eve}\|^2} \right) h_{Eve}
    \end{equation}

    This can be expressed using the projection matrix onto Eve's null space:
    \begin{equation} \label{eq:equation12}
        w = \left( I - \frac{h_{Eve} h_{Eve}^H}{\|h_{Eve}\|^2} \right) h_{Bob}
    \end{equation}


    \subsubsection*{Calculation of Correlation $\rho$}
    Before calculating $\alpha$, we need $\rho$.
    This decomposes into the product of the elevation ($G_z$) and azimuth ($G_y$) array gains.
    \begin{equation} \label{eq:equation13}
        |\rho| \approx \frac{1}{N_{total}} (G_z \cdot G_y)
    \end{equation}

    \begin{itemize}
        \item \textbf{Elevation Gain ($G_z$):} The angular separation is small ($\Delta \phi \approx 1^\circ$). The beams overlap constructively, so $G_z \approx N_z = 8$.
        \item \textbf{Azimuth Gain ($G_y$):} The users are widely separated in azimuth ($\approx 52^\circ$). The side-lobe leakage is low, approximated by the Dirichlet kernel as $G_y \approx 1$.
    \end{itemize}

    \begin{equation} \label{eq:equation14}
        |\rho| \approx \frac{8 \cdot 1}{64} = 0.125
    \end{equation}

    \subsection*{Array Normalisation Factor ($\alpha$)}
    To normalise the received signal to magnitude 1 (Unity gain), we enforce the constraint $w^{H} h_{Bob} = 1$.
    Let the final vector be $w_{final} = \alpha w$.

    Substituting the expression for $w$ into the constraint:
    \begin{equation} \label{eq:equation15}
        \alpha \left[ \left( I - \frac{h_{Eve} h_{Eve}^H}{\|h_{Eve}\|^2} \right) h_{Bob} \right]^H h_{Bob} = 1
    \end{equation}

    Expanding the inner term:
    \begin{equation} \label{eq:equation16}
        \alpha \left( h_{Bob}^H h_{Bob} - \frac{h_{Bob}^H h_{Eve} h_{Eve}^H h_{Bob}}{\|h_{Eve}\|^2} \right) = 1
    \end{equation}

    Recognising that $h_{Bob}^H h_{Eve} h_{Eve}^H h_{Bob} = |h_{Eve}^H h_{Bob}|^2$:
    \begin{equation} \label{eq:equation17}
        \alpha \left( \|h_{Bob}\|^2 - \frac{|h_{Eve}^H h_{Bob}|^2}{\|h_{Eve}\|^2} \right) = 1
    \end{equation}

    Factorising $\|h_{Bob}\|^2$:
    \begin{equation} \label{eq:equation18}
        \alpha \|h_{Bob}\|^2 \left( 1 - \frac{|h_{Eve}^H h_{Bob}|^2}{\|h_{Eve}\|^2 \|h_{Bob}\|^2} \right) = 1
    \end{equation}

    The term in the fraction is exactly the square of the correlation coefficient $|\rho|^2$. Thus:
    \begin{equation} \label{eq:equation19}
        \alpha = \frac{1}{\|h_{Bob}\|^2 (1 - |\rho|^2)}
    \end{equation}

    Substituting $|\rho| \approx 0.125$ into the $\alpha$ equation:
    \begin{equation} \label{eq:equation20}
        \alpha = \frac{1}{64 (1 - 0.125^2)} = \frac{1}{64(0.984)} \approx 0.0159
    \end{equation}

    \subsection*{Analytic Derivation of $\beta$}
    The final beamforming equation is:
    \begin{equation} \label{eq:equation21}
        w_{final} = \alpha h_{Bob} - \underbrace{\left( \alpha \frac{h_{Eve}^H h_{Bob}}{\|h_{Eve}\|^2} \right)}_{\beta} h_{Eve}
    \end{equation}

    To understand why $\beta$ is complex, we expand the inner product $h_{Eve}^H h_{Bob}$. This is a sum of phasors. For a uniform linear array (ULA) of $N$ elements with spatial frequency difference $\Delta \psi$:
    \begin{equation} \label{eq:equation22}
        \text{Inner Product} = \sum_{n=0}^{N-1} e^{j n \Delta \psi} = \underbrace{e^{j \frac{N-1}{2} \Delta \psi}}_{\text{Phase Rotation}} \cdot \underbrace{\frac{\sin(N \Delta \psi / 2)}{\sin(\Delta \psi / 2)}}_{\text{Real Magnitude (Dirichlet)}}
    \end{equation}

    The term $e^{j \frac{N-1}{2} \Delta \psi}$ represents the \textbf{phase shift} due to the array center being at index $(N-1)/2$ rather than 0.

    \textbf{Applying to our parameters:}
    \begin{itemize}
        \item The Magnitude part corresponds to our estimated gains ($G_z \approx 8, G_y \approx 1$).
        \item The Phase Rotation comes from the Azimuth difference $\Delta_y \approx 0.862$.
        \item Phase Argument $\Phi \approx \frac{7}{2} \pi (0.862) \approx 3.017 \pi \equiv 0.017 \pi$ (modulo $2\pi$).
        \item This small residual phase angle rotates the vector slightly off the real axis.
    \end{itemize}

    \textbf{Calculation:}
    \begin{equation} \label{eq:equation23}
    \begin{split}
        \beta &= \frac{\alpha}{N_{total}} (8 \cdot e^{-j 0.26}) \quad \text{(approximate phase from geometry)} \\
        \beta &= \frac{0.0159}{64} (8) (\cos(-0.26) + j\sin(-0.26)) \\
        \beta &\approx 0.00198 (0.96 - j0.25) \\
        \beta &\approx 0.0019 - j0.0005
    \end{split}
    \end{equation}

    The imaginary component $-j0.0005$ arises strictly from this phase rotation term in the geometric series summation.

    \subsection*{Final Calculation for Antenna 1}
    \underline{For Antenna 1 (index 0):}
    Assuming reference elements $[h_{Bob}]_0 = 1$ and $[h_{Eve}]_0 = 1$:
    \begin{equation} \label{eq:equation24}
    \begin{split}
        w_0 &= \alpha [h_{Bob}]_0 - \beta [h_{Eve}]_0 \\
        w_0 &= 0.0159(1) - (0.0019 - j0.0005)(1) \\
        w_0 &= 0.0140 + j0.0005
    \end{split}
    \end{equation}

    This calculation is then repeated for all 64 antenna elements to construct the full beamforming vector $w$.

\end{document}
